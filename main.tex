\documentclass{utsignal}

\usepackage{amsmath}
\usepackage{amssymb}
\usepackage{wrapfig}
\usepackage{verbatim}
\usepackage{fancyvrb}
\usepackage{lscape}
\usepackage{rotating}
\usepackage{xepersian}
\usepackage{listings}
\usepackage{color}
\usepackage[utf8]{inputenc}
\usepackage{csquotes}

\title{پروژه - فاز ۲}
\course{سیگنال‌ها و سیستم‌ها}
\author{\href{mailto:h.barkhordarpour@ut.ac.ir?subject=[SS\%20S98 A2]}{هدی برخوردارپور}، 
\href{mailto:ranjbar.ali@ut.ac.ir?subject=[SS\%20S98 A2]\%20}{علی رنجبر}}
%\lecturer{امیرمسعود ربیعی}
\deadline{سه‌شنبه ۲۱ خرداد ۱۳۹۸، ساعت ۲۳:۵۵}
\graphicspath{{./images/}}


\begin{document}
	\maketitle
	\section*{مقدمه و یادآوری فاز ۱}
	با انجام فاز ۱ باید کدی نوشته باشید که با ورودی گرفتن یک فایل صوتی، ماتریس مربوط به جفت پیک‌های آن را برمی‌گرداند. این ماتریس در حالت کلی به شکل زیر است که در هر سطر آن اطلاعات زمانی و فرکانسی مربوط به هر جفت پیک ذخیره شده است.
	\begin{center}
		\begin{latin}
			\begin{tabular}{|c|c|c|c|}
				$t_1$ & $t_2$ & $f_1$ & $f_2$ \\
				\vdots&\vdots&\vdots&\vdots\\
				$t_j$ & $t_k$ & $f_j$ & $f_k$ \\
				\vdots&\vdots&\vdots&\vdots\\
				$t_m$ & $t_n$ & $f_m$ & $f_n$ \\
			\end{tabular}
		\end{latin}
	\end{center}
	حال می‌خواهیم تابع \lstinline[language=Matlab]{make_table} را با ورودی این ماتریس اجرا کنیم تاجدول مربوط به آهنگ را ساخته و در پایگاه داده ذخیره کنیم. جدول مربوط به هر آهنگ به شکل زیر خواهد بود:
	\begin{center}
		\begin{latin}
			\begin{tabular}{|c|c|c|c|}
				$f_1$ & $f_2$ & $t_1^s$ & $t_2^s-t_1^s$\\
				&&&\vdots\\
				$f_j$ & $f_k$ & $t_j^s$ & $t_k^s-t_j^s$\\
				&&&\vdots\\
				$f_p$ & $f_q$ & $t_p^s$ & $t_q^s-t_p^s$\\
			\end{tabular}
		\end{latin}
	\end{center}
	اگر یک کلیپ صوتی از آهنگ‌های موجود در پایگاه داده داشته باشیم، هر سطر جدول کلیپ باید سطری متناظر در جدول آهنگ داشته باشد. برای مثال اگر $(f_1, f_2, t_1^c, t_2^c-t_1^c)$ یک سطر در جدول کلیپ و $(g_1, g_2, t_1^s, t_2^s-t_1^s)$ سطر متناظر آن در جدول آهنگ باشد، باید $f_1=g_1$، $f_2=g_2$ و $t_2^c-t_1^c=t_2^s-t_1^s$ باشند. چون این سه مقدار با شیفت زمانی کلیپ نسبت به ابتدای آهنگ، تغییر نمی‌کنند. در نتیجه استفاده از سه‌تایی $(f_1, f_2, t_2^c-t_1^c)$ راه خوبی برای جستوجوی تطبیق در پایگاه داده است.
	
	\section{ساخت پایگاه داده}
	در این قسمت دو پایگاه داده را برای ذخیره‌ی لیست آهنگ‌ها و اطلاعات زمانی جفت پیک‌ها آماده می‌کنیم. برای ذخیره‌ی لیست آهنگ‌ها کافیست از یک بردار تک ستون استفاده کنید که در هر ستون اسم فایل آهنگ نوشته شده است. شماره‌ی ردیف هر آهنگ شناسه‌ی\LTRfootnote{ID} آن آهنگ می‌باشد. این پایگاه داده را در فایل \lr{SONGID\_DB.mat} ذخیره کنید. پایگاه داده دوم  باید دو ستون داشته باشد. تعداد ردیف‌ها و نحوه‌ی پر کردن آن را در ادامه توضیح می‌دهیم. این پایگاه داده را در فایل \lr{HASHTABLE.mat} ذخیره کنید. 
	
	برای هر آهنگ دو فایل \lr{SONGID\_DB.mat} و \lr{HASHTABLE.mat} را بارگذاری کنید. ابتدا پایگاه داده‌ی \lr{SONGID\_DB} را به‌روز کنید. تابع \lstinline[language=Matlab]{peak_to_pair} چهارتایی $(t_1, t_2, f_1, f_2)$ برای تمام جفت پیک‌های یک آهنگ را تولید می‌کند. برای هر کدام از این چهار‌تایی‌ها ابتدا با استفاده از تابع چکیده‌ساز اندیس 
	\lr{(index = $h(f_1^c, f_2^c, t_2^c-t_1^c)$)} مناسب آن را پیدا کرده سپس در ردیف آن اندیس پایگاه داده \lr{HASHTABLE} شناسه‌ی آهنگ و مقدار $(t_1^s-t_1^c)$ را ذخیره کنید.
\end{document}
